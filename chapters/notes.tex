
% dit zijn notities, dat houden we kort want we hebben de andere versie om er een bende van te maken!


% belangrijk om in de gaten te houden:
%   - je schrijft dit verslag om als naslag werk te dienen voor anderen, modelwerk en        experimenten moeten dus herhaalbaar zijn en daar moet je dus compleet en grondig       mee omgaan!




% eerst een inleiding, die kan ook achteraf beschreven worden, de preface
% - de introductie heeft de volgende opbouw:
% - beginnen met de relevantie, waarom is wat je gedaan hebt relevant?
% - ingaan op de achtergrond: waarom verdiepen mensen zich in kikker en dergelijke en wat is er allemaal over bekent, hier kunn je ook ingaan op de morphologie van de boomkikker
% vervolgens het doel van het onderzoek en globaar de outline: grove beschrijving van hoe de inhoud opgezet is om naar dat doel toe te werken

% - De opzet is dat we toewerken van de situatie zoals we die in de natuur tegenkomen naar een discreet model dat zo goed als het gaat de eigenschappen heeft van de situatie die we in de natuur tegen komen.
% - Gebaseerd op de tekortkomingen van het model zoals we dat natuurgetrouw gemodelleerd hebben kunnen we toewerken naar het HGO model. 
% - de resultaten van de modellen kunnen daarna besproken worden, hier kan dan ook weer onderscheid gemaakt worden tussen de discrete modellen en het HGO model
% - Na de gemodelleerde resulaten komt het experiment. Beschrijven als volgt:
%        - Wat is het doel van het experiment, hiermee teruggrijpen op de resultaten van            de modellen die in de sectie ervoor beschreven zijn
%        - Wat is de opzet van het experiment en hoe is het uitgevoerd
%        - wat zijn de resulaten
% vervolgens discussie over:
%   - de aannames voor alle modellen, is het HGO model geschikt als vervanging voor het       discrete model?
%   - Keuzes en aannames van de proefopstelling: komt de proefopstelling overeen met         wat ik gemodelleerd heb, zijn er verder nog aannames gemaakt of randvoorwaarden        te benoemen waar wat over te zeggen is?
%   - De intepretatie van de gemodelleerde resulaten en de vergelijking van deze             resultaten met de resultaten van het experiment dat ik uitgevoerd heb  
%   